%!TEX root = main.tex
\author{Autor: An-Nam Pham}
\section{Tugendethik}
In diesem Kapitel wird die Tugendethik auf die Firma Venus (im Folgenden kurz \textit{Venus} genannt) angewendet. Dafür wird untersucht, ob die Kardinaltugenden nach Aristoteles von Venus erfüllt werden. Im Umfang dieser Arbeit beschränken sich die Kardinaltugende auf Gerechtigkeit und Tapferkeit. Zusätzlich dazu wird untersucht, inwiefern Venus die Tugend der Gesellschaft beeinflusst.
\subsection{Gerechtigkeit}
Es wird nun untersucht, ob Venus die Tugend Gerechtigkeit (\textit{dikaiosýne}) nach Aristoteles erfüllt. Zur Gerechtigkeit sagt Aristoteles in seiner Nikomachische Ethik (1133b9) folgendes:
\begin{quote}
\textit{\glqq Es ergibt sich daraus, dass das gerechte Handeln die Mitte ist zwischen dem Unrechttun und dem Unrechtleiden.\grqq} 
\end{quote}
Es muss also auch betrachtet werden, ob Venus Unrecht tut oder Unrecht leidet.\\
Aristoteles teilt die Gerechtigkeit in zwei Unterbegriffe auf:
\begin{itemize}
\item Austeilende Gerechtigkeit
\item Ausgleichende Gerechtigkeit
\end{itemize}
In das Soziale Netzwerk Venus kann sich jeder bewerben und wird dann von den Mitgliedern dieses Netzwerks beurteilt. Dies geschieht auf Grundlage eines Fotos und einer kurzen Selbstbeschreibung. Somit wird nicht bekannt, wie viel Aufwand und Mühe hinter einer Bewerbung steckt. Dadurch kann es keine austeilende Gerechtigkeit geben, da die Bewerber unabhängig von ihrer vorherigen Vorbereitung beurteilt werden.

Dass in Venus keine austeilende Gerechtigkeit herrscht, ist nicht unfair oder unethisch. Von Prüfungen (z.B. Klausuren) ist das gleiche Vorgehen bekannt: Unabhängig davon, wie viel Vorbereitung und Mühe ein Student in die Klausurvorbereitung steckt, kann er bei schlechter Leistung in einer Klausur durchfallen, während andere weniger gelernt haben und eine gute Note schreiben.

Aus der Sicht der ausgleichenden Gerechtigkeit erfüllt Venus allerdings die Bedingungen:\\
Dafür dass die Mitglieder einen monatlichen Beitrag zahlen, bekommen sie das Recht selber entscheiden zu können, wen sie in ihrer \glqq Gesellschaft\grqq~haben wollen und wen nicht. Und wenn sie nur Leute haben wollen, die sie selber schön finden, dann ist das ethisch korrekt, da sie dafür bezahlt haben dies tun zu können.

Betrachtet man Venus nun aus Sicht der Gerechtigkeit nach modernen Konzeptionen, muss untersucht werden, ob sie die Bedingungen der Verfahrensgerechtigkeit, Verteilungsgerechtigkeit oder der interaktionalen Gerechtigkeit erfüllt.

Die Verteilungsgerechtigkeit kann hier mit der austeilenden Gerechtigkeit nach Aristoteles gleichgesetzt werden. Daraus folgt, dass in Venus keine Verteilungsgerechtigkeit herrscht.\\
In Venus herrscht dafür die Verfahrensgerechtigkeit, da hier im Bewerbungsverfahren eine hohe Transparenz gegeben ist. Dadurch kann jeder Bewerber sich vorher informieren, worauf bei der Bewerbung geachtet wird. Außerdem kann er sich bei Ablehnung eine Rückmeldung geben lassen, warum er abgelehnt wurde. Der Bewerber weiß somit jederzeit, warum er wie beurteilt wurde.

Auch die Interaktionale Gerechtigkeit ist bei Venus gegeben. Dadurch, dass die Firma den Grundsatz vertritt, dass jede Person mit Respekt, ehrlich, höflich und angemessen behandelt wird, handelt Venus fair. Außerdem werden bei Entscheidungen, die Mitgleider oder Bewerber betreffen, die Betroffenen mitberücksichtigt und wahrheitsgemäß informiert.

Insgesamt betrachtet kann gesagt werden, dass Venus weder was Unrechtes tut noch Unrecht leidet.

\subsection{Tapferkeit}
In diesem Unterkapitel wird untersucht, inwiefern die Gründer der Firma Venus tapfer sind.
Nach der Mesoteslehre ist die Tapferkeit die Mitte zwischen den Extremen Feigheit und Tollkühn. Diese Mitte kann nur gefunden werden, wenn vorher bereits Erfahrung mit den beiden Extremen gemacht wurden. Laut der Gründungsgeschichte von Venus haben die Gründer Erfahrungen mit beiden Extremen gemacht.

Am Anfang hatten die Gründer eine Idee: Möglichst viele Menschen zu motivieren, ein erfülltes Leben zu haben. Bei der Ausarbeitung ihrer Idee ist ihnen bewusst geworden, dass ganz viel Kritik von außen kommen wird. Ihnen war bewusst, dass Feigheit ihnen den Weg zur Verwirklichung ihrer Idee versperren würde. Und tollkühn wollten sie auch nicht sein. Ihrer Meinung nach sind Firmen wie Abercrombie \& Fitch oder Hollister tollkühn. Diese Firmen stürzen sich auf den Markt und sagen, dass ihre Kleidung nur was für die hübschen und coolen Leute sind. Da Venus aber Menschen motiveren möchte, ein erfülltes Leben zu haben, mussten sie die Mitte zwischen Feigheit und Tollkühn finden und tapfer gegen Kritiken und Ablehnungen kämpfen bis Venus ihre ersten Erfolge zeigt.\\
Diese Herangehensweise zeigt die Tapferkeit in dem Vorgehen der Firma Venus.

\subsection{Tugendethik und das Ziel von Venus}
Im Alltag werden Menschen unzählig oft vor Entscheidungen gestellt. Sei es bewusst oder unbewusst. Die große Frage lautet nun, wie bei einem Entscheidungsproblem moralisch korrekt entschieden wird. Das Empfinden von \textit{richtig} oder \textit{falsch} ist hierbei nach der modernen Tugendethik vom gesellschaftlichem Rahmen abhängig. Die Erfahrungen, die in einer umgebenen Gesellschaft gemacht werden, die Erziehung, die man in seiner Umgebung (Eltern, Lehrer) genossen hat und die beigebrachten Werte und Regeln der Gesellschaft bilden die Grundlage für das moralische Empfinden einer Person. Aufgrund gesellschaftlicher Normen bildet eine Person auch eine Vorstellung vom Schönheitsideal und einem gewünschten Erscheinungsbild.

Erfüllt eine Person nicht dem Schönheitsideal oder einem gewünschten Erscheinungsbild, kann er von der Gesellschaft abgelehnt werden, was zu einer Senkung des Selbstvertrauens führt. Dadurch kann er keine Glückseligkeit oder Eudaimonie (nach Aristoteles) mehr erreichen. Nach Aristoteles ist es aber das höchste Gut, Eudaimonie zu erreichen, was nur durch gut leben und sich gut verhalten bzw. tugendlich leben und tugendlich verhalten geht. Und genau das versucht Venus zu erreichen. Durch das soziale Netzwerk sollen Menschen dazu motiviert werden, dem gewünschten Erscheinungsbild und Schönheitsideal der Gesellschaft zu entsprechen. Dadurch werden sie von der Gesellschaft anerkannt und können ihr Selbstbewusstsein steigern, was einem Eudaimonie näher bringt.

Die Erfahrung der Menschen zeigt: Selbstbewusste Menschen können in eine Menschenmenge gehen, sich bewusst an Konversationen beteiligen und ihre Vorhaben erfolgreich umsetzen. Im Gegensatz dazu können Menschen mit niedrigem Selbstbewusstsein nur schwer erfolgreich handeln, da ihre Unsicherheit im Umgang mit der Gesellschaft sie daran hindert.

Das Selbstbewusstsein von Menschen mit niedrigem Selbstbewusstsein kann bereits durch Anpassung ihrer Kleidung und Aufmachung bzw. \glqq Style\grqq~an die gesellschaftliche Norm gesteigert werden.
Viele selbstbewusste Menschen achten sehr auf ihr Erscheinungsbild. Sie gehen ins Fitness-Studio, machen Sport, ernähren sich gesund und achten auf gute Kleidung, um der gesellschaftlichen Norm zu entsprechen und anerkannt zu werden.\\
Bereits im Kindesalter erziehen deswegen viele Eltern ihre Kinder dazu aufrecht zu stehen oder immer zu lächeln. Das hilft das Ansehen der Kinder in der Gesellschaft zu steigern und steigert das Selbstbewusstsein der Kinder.

Venus möchte Menschen dazu motivieren genau dasselbe zu tun. Wird man bei der Bewerbung abgelehnt, kann man die Gründe dafür erfahren und an sich arbeiten.\\
Böse Zungen könnten nun behaupten, dass Venus Menschen in hässlich und schön kategorisiert und das Bedürfnis weckt, Schönheitsoperationen an sich durchführen zu lassen.

Zum ersten Punkt kann gesagt werden, dass Venus nicht in hässlich oder schön kategorisiert, sondern in Erscheinungsbild gesellschaftlich akzeptiert oder nicht akzeptiert. Und wer nicht akzeptiert wird, kann durch Venus Unterstützung bekommen, um an sich zu arbeiten. Gemäß dem Spruch: \glqq Aus Fehlern lernt man\grqq.

Zum zweiten Punkt muss überlegt werden, ob Schönheitsoperationen moralisch verwerflich sind.
Wenn Schönheitsoperationen Menschen mit niedrigem Selbstbewusstsein helfen, ihr Selbstbewusstsein zu steigern, steigert sich dadurch auch ihr Glück und sie nähern sich Eudaimonie. Solange die Schönheitsoperationen nur das Selbstbewusstsein stärken sollen ohne dem Körper zu schaden, kann dies nicht negativ verurteilt werden. Schließlich ist Selbstbewusstsein eine Voraussetzung für ein erfolgreiches Leben  in öffentlicher und privater Hinsicht.\\
In diesem Fall kann eine Schönheitsoperation auch mit einem körperlichem Trainining oder Make-Up und Styling gleichgesetzt werden. Denn all das hat ein Ziel: Selbstbewusstsein stärken und glücklicher zu werden. Und das soziale Netzwerk Venus kann das fördern.