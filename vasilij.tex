% -*- LaTeX-command: pdflatex; TeX-master: "main.tex"; -*-
\author{Autor: Vasilij Schneidermann}
\section{Pflichtenethik}

Bei der Pflichtenethik, auch als Deonthologie bekannt, wird besonderes
Augenmerk auf der \emph{Pflichterfüllung} gelegt.  Grundlegende
Annahme dabei ist dass sich aus der vorhandenen Vernunft des Menschen
guter Wille und Absicht ergeben welche in Form von freiwillig
selbstaufgelegten Normen umgesetzt werden.  Neben rechtsverbindlichen
Normen gibt es dabei noch ethische Normen denen jedes Individuum für
sich nachgeht.  Kant führt dafür als Grundprinzip den
\emph{Kategorischen Imperativ} ein mit welchem man prüfen kann ob eine
gegebene Norm ethisch ist.

Es stellt sich die Frage ob es für Venus überhaupt möglich ist nach
Kant moralisch zu handeln.  Zur Beurteilung werden die zwei
bekanntesten Formen des kategorischen Imperativs hinzugezogen, die
\emph{Universalisierungsformel} und die \emph{Selbstzweckformel}.
Folgende \emph{Maximen} werden genauer untersucht:

\begin{enumerate}
\item Motiviere möglichst viele Menschen dazu ein erfülltes Leben zu
  haben.
\item Akzeptiere nur schöne Menschen.
\end{enumerate}

\subsection{Universalisierungsformel}

\begin{quote}
,,Handle nur nach derjenigen Maxime, durch die du zugleich wollen kannst, dass sie ein allgemeines Gesetz werde.''

"= Immanuel \textsc{Kant}: AA IV, 421
\end{quote}

Hinter der ersten Maxime ist ohne Zweifel Philanthropie zu erkennen.
Zwar ist es für Kritiker möglich einzuwerfen dass das Ziel nach
Quantität zu gehen eindeutig wirtschaftlich motiviert ist, jedoch
überwiegen aufgrund der Tatsache dass es sich um ein soziales Netzwerk
handelt die Netzwerkeffekte und damit der Nutzen für alle dadurch
erreichten Menschen.  Die Frage ob man möchte dass diese Maxime Gesetz
werde ist deswegen zu bejahen.

Bei der zweiten Maxime ist der dahinterliegende Wille schwieriger zu
erkennen.  Es ist kein Geheimnis, dass Menschen seit Anbeginn der
Zeiten attraktive Menschen bevorzugen\footnote{Vgl. Homers Ilias},
aber rechtfertigt dies eine Auslegung als allgemeingültiges Gesetz?
Um dieses Dilemma aufzulösen ziehen wir die Beobachtung hinzu, dass
unser soziales Netzwerk das echte Leben mit seinen ungeschriebenen
Gesetzen wiederspiegelt.  Deswegen stellt es für Venus kein Problem
dar die Bevorzugung attraktiver Menschen als Gesetz für sich
aufzunehmen.  Für die Allgemeinheit hingegen kann man das so nicht
behaupten.  Aufgrund dieses Widerspruchs ist die zweite Maxime nach
Kant nicht vertretbar.

\subsection{Selbstzweckformel}

\begin{quote}

,,Handle so, dass du die Menschheit sowohl in deiner Person, als in der Person eines jeden anderen jederzeit zugleich als Zweck, niemals bloß als Mittel brauchst.''

"= Immanuel \textsc{Kant}: AA IV, 429[9]

\end{quote}

Die erste Maxime wird dieser Definition zufolge erfüllt, da Venus im
Sinne von der Menschheit für die Menschheit handelt.  Zudem erreichen
wir durch den Netzwerkeffekt weitaus mehr Menschen als wir in unserem
sozialen Netzwerk haben.

Die zweite Maxime erfordert etwas mehr Durchleuchtung.  Einerseits
kann man dafür argumentieren, dass Venus schöne Menschen als Mittel
für sich verwendet, andererseits ist dabei auch ein Zweck erkennbar,
nämlich die Erfüllung des Versprechens an die Mitglieder ein für sie
erwünschenswertes soziales Netzwerk anzubieten.  Jedoch ist wie auch
bei der Universalisierungsformel erkennbar dass die Anwendung dieser
Argumentation außerhalb des sozialen Netzwerks nicht befürwortbar
ist.  Daher kann auch sie nicht vertreten werden.

\subsection{Fazit}

Durch die Anwendung des kategorischen Imperativs wird die erste Maxime
als wünschenswert und die zweite als nicht mit dem Wohl für die
Allgemeinheit vereinbar beschrieben werden.
